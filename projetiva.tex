\documentclass{scrbook}
\KOMAoptions{
	fontsize = 11pt,
	parskip = half,
	paper = a4,
	twoside = false,
}
\usepackage[brazilian]{babel}	% Traduz as coisas automáticas (como datas, capítulos, etc) para PT-BR.
\usepackage{csquotes}
\usepackage{enumitem}			% Cria listas maneiras.
\usepackage{amsmath, amssymb, amsfonts}	% Packages da American Mathematical Society.
 
\usepackage{biblatex}
\addbibresource{referencias.bib}

% Definição de ambientes de teoremas.
\usepackage{amsthm}
\newtheorem{thm}{Teorema}
\newtheorem{prop}[thm]{Proposição}
\newtheorem{exer}{Exercício}
\newtheorem*{que}{Questão}
\theoremstyle{definition}
\newtheorem{defn}{Definição}
\newtheorem{pdefn}[defn]{Proposta de Definição}
\newtheorem{exmp}{Exemplo}
\newtheorem{lem}[thm]{Lema}
\newtheorem{cor}[thm]{Corolário}

\newcommand{\RR}{\mathbb{R} }
\newcommand{\KK}{\mathbb{K} }
\newcommand{\ZZ}{\mathbb{Z} }
\newcommand{\PP}{\mathbb{P} }

\title{Geometria no Plano Projetivo}
\author{}
\date{}

% O seguinte comando diz quais "\include"s levar a sério, e quais desconsiderar.
% É útil mexer nisso para o TeX não precisar compilar o arquivo completo, caso você esteja mexendo em somente um capítulo.
%\includeonly{}

\begin{document}
	\maketitle

	\tableofcontents

	\chapter*{Preâmbulo}

Este livro de geometria projetiva é um compilado do aprendizado dos autores em diversos cursos de geometria projetiva lecionados pelo Prof. Luciano Guimarães Monteiro de Castro, tanto na Academia Matematicamente, quanto na Escola Eleva. É uma tentativa de continuação do artigo \cite{castro2000}.

O foco desse livro é explorar as aplicações de geometria projetiva sobre espaços reais para resolver problemas de olimpíadas, para isso é necessário que o leitor tenha alguma familiaridade com álgebra linear.


	\part{Abordagem Axiomática}

	\chapter{A reta real não tem harmonia}

\section{Divisão Harmônica}

\begin{exer}
  Dados dois pontos \(A\) e \(B\) distintos na reta real, marque, com régua e compasso, todos os pontos \(P\) na reta real tais que \(\frac{AP}{BP} = 2\).
\end{exer}

Você observará que existem dois pontos, digamos \(P\) e \(Q\) --- um dentro do segmento \(AB\) e o outro fora do segmento \(AB\) ---  tais que \(\frac{AP}{BP} = \frac{AQ}{BQ} = 2\).

Por terem essa propriedade, dizemos que \(P\) e \(Q\) dividem \(A\) e \(B\) harmonicamente, na razão \(2\).

% Insira a figura A---P-------B-----------Q.

Não há nada especial com o número \(2\); \emph{parece} que, dado qualquer \(\lambda\) real positivo, existem sempre dois pontos que dividem \(A\) e \(B\) harmonicamente na razão \(\lambda\)...
Na verdade, tem um valor de \(\lambda\) para o qual isso não é verdade.

\begin{exer}
  Dados dois pontos \(A\) e \(B\) distintos na reta real, marque, com régua e compasso, todos os pontos \(P\) na reta real tais que \(\frac{AP}{BP} = 1\).
\end{exer}

Você observará que somente um ponto possui a propriedade acima; o ponto médio do segmento \(AB\).
Não existe ponto fora do segmento \(AB\) com essa propriedade --- de fato, um dos segmentos \(AP\) ou \(BP\) estará contido no outro, e portanto um será menor que o outro.


	\chapter{Definindo o Plano Projetivo}

\section{Plano Euclidiano}

Você talvez esteja acostumado com a noção do Plano Euclidiano.
Essa estrutura envolve muitos conceitos, incluindo (mas não limitado a):
\begin{enumerate}[label = \textbullet]
	\item pontos,
	\item retas,
	\item segmentos de reta,
	\item distância,
	\item ângulos,
	\item polígonos (triângulos, quadriláteros, \dots),
	\item circunferências,
	%\item conics (ellipses, parabolas, and hyperbolas),
	\item transformações (rotações, translações, reflexões, \dots).
\end{enumerate}

Apesar de todas esses conceitos estarem empacotados no Plano Euclidiano, parece que os conceitos mais fundamentais são os primeiros dessa lista.
Em outras palavras, os conceitos de \emph{pontos} e \emph{retas} parecem ser mais simples e mais relacionados à definição de Plano Euclidiano, do que os conceitos de \emph{ângulos} e \emph{transformações}.

Nós poderíamos gastar algumas folhas formalizando a definição de Plano Euclidiano (usando o que os matemáticos chamam de \emph{axiomas}) e nos converncer de que, de fato, \emph{pontos} e \emph{retas} são intimamente relacionados ao que realmente o Plano Euclidiano é.
Porém, nós prefirimos não fazer isso.
Na verdade, vamos listar duas propriedades que o Plano Euclidiano \emph{parece} ter:
\begin{enumerate}[label = (\textit{\roman*})]
	\item Para cada par de pontos distindos, existe uma única reta passando por ambos. \label{item:pe:i}
	\item Para cada par de retas distintas, existe um único ponto que está em ambas. \label{item:pe:ii}
\end{enumerate}

\begin{exer} \label{exer:propriedadesdoplanoeuclidiano}
	Descubra se as propriedades acima são verdades ou não no Plano Euclideano.
\end{exer}

\section{Plano Projetivo}

No Exercício \ref{exer:propriedadesdoplanoeuclidiano}, você talvez tenha notado que, embora a propriedade \ref{item:pe:i} seja verdadeira, \ref{item:pe:ii} não é.
Existe até um nome especial para um par de retas que não se encontra em nenhum ponto: dizemos que essas retas são \emph{retas paralelas}.

% Aside: We could here define a relation between pair of lines, namely "line \(\ell_1\) is parallel to line \(\ell_2\)",
% and deduce that this relation is an equivalence relation; thus generates equivalence classes.

%In fact, one of the axioms of the Euclidean Plane directly contradicts \ref{item:pe:ii}:
%\begin{quote} % www.math.brown.edu/tbanchof/Beyond3d/chapter9/section01.html 
%	For every given point not on a given line, there exists a unique line passing through the given point that does not meet the given line.
%\end{quote}

Mas... as propriedades \ref{item:pe:i} e \ref{item:pe:ii} são tão bonitas (pelo menos pra nós).
É tão triste que elas não são verdadeiras.
Bem, nós somos matemáticos!
Nós podemos fazer o que quisermos com nossas definições (apesar de ter que viver com suas consequências).

\begin{pdefn} \label{pdefn:planoprojetivo}
	Um \emph{plano projetivo} é uma estrutura que consiste em três coisas:
	\begin{enumerate}[label = \textbullet]
		\item um conjunto \(\mathcal P\) de pontos,
		\item um conjunto \(\mathcal L\) de retas, e
		\item uma noção de incidência, i.e., se \(P\) é um ponto e \(\ell\) é uma reta, nós poderemos dizer se \(\ell\) passa (ou não) pelo ponto \(P\).
	\end{enumerate}

	Adicionalmente, precisamos das seguintes propriedades:
	\begin{enumerate}[label = (\textit{\roman*})]
		\item Para cada par de pontos distindos, existe uma única reta passando por ambos. \label{item:pp:i}
		\item Para cada par de retas distintas, existe um único ponto que está em ambas. \label{item:pp:ii}
	\end{enumerate}
\end{pdefn}

Essa proposta de definição é muito boa, porém ela aceita noções de planos projetivos bastante distintos do plano euclidiano.

\begin{exmp}
	Considere \(\mathcal P\) um conjunto qualquer de pontos, e \(\mathcal L = \{\ell\}\) um conjunto contendo uma única reta \(\ell\).
	Defina a noção de incidência de modo que qualquer ponto esteja na reta \(\ell\).
	Ambas as propriedades \ref{item:pp:i} e \ref{item:pp:ii} são satisfeitas; portanto essa estrutura é um plano projetivo pela Definição \ref{pdefn:planoprojetivo}.
\end{exmp}

\begin{exmp}[Plano de Fano]
	Seja \(\mathcal P = \{0, 1, 2, 3, 4, 5, 6\}\) e \(\mathcal L = \{\ell_0, \ell_1, \ell_2, \ell_3, \ell_4, \ell_5, \ell_6\}\). Defina a noção de incidência de modo que
	\begin{align*}
		\ell_0 &= \{0, 1, 3\} \\
		\ell_1 &= \{1, 2, 4\} \\
		\ell_2 &= \{2, 3, 5\} \\
		\ell_3 &= \{3, 4, 6\} \\
		\ell_4 &= \{0, 4, 5\} \\
		\ell_5 &= \{1, 5, 6\} \\
		\ell_6 &= \{0, 2, 6\}.
	\end{align*}
	
	Você consegue verificar que ambas as propriedades \ref{item:pp:i} e \ref{item:pp:ii} são satisfeitas; portanto essa estrutura é um plano projetivo pela Definição \ref{pdefn:planoprojetivo}.
\end{exmp}

\subsection{Plano Projetivo Real}

Considere \(\mathcal P\) consistindo de todos os pontos do plano euclidiano; vamos também adicionar um ponto (que não está no plano euclidiano) para cada feixe de retas paralelas do plano euclidiano; isto é, além dos pontos convencionais do ponto euclidiano, para cada direção, existe um ponto novo --- que chamaremos de ponto no infinito relacionado a essa direção.

Considere \(\mathcal L\) consistindo de todas as retas do plano euclidiano, com a diferença que essa reta também irá incidir no ponto do infinito relacionado a sua direção. Além disso, também adicionaremos uma reta em \(\mathcal L\) que contém todos os pontos do infinito --- que chamaremos de reta do infinito.

Apesar de ser confuso, vamos tentar verificar as condições \ref{item:pp:i} e \ref{item:pp:ii}.

Para a condição \ref{item:pp:i}, vamos dividir nas possibilidades para os pares de pontos.

\begin{enumerate}[label = \textbullet]
	\item Dados dois pontos \(P\) e \(Q\) do plano euclidiano, sabemos que há uma única reta que passa por eles no plano euclidiano; a mesma reta passará por eles na nova estrutura proposta.
	\item Dado um ponto \(P\) do plano euclidiano, e um ponto \(Q_\infty\) do infinito, sabemos que há uma única reta que passa por \(P\) e está na direção relacionada ao ponto \(Q_\infty\). A reta análoga a esta passará por \(P\) e \(Q_\infty\) na nova estrutura proposta.
	\item Dados dois pontos \(P_\infty\) e \(Q_\infty\) do infinito, a chamada reta do infinito é a única reta que passa por ambos na estrutura proposta.
\end{enumerate}

Para a condição \ref{item:pp:ii}, vamos dividir nas possibilidades para os pares de retas.

\begin{enumerate}[label = \textbullet]
	\item Dadas duas retas \(\ell_1\) e \(\ell_2\) concorrentes no plano euclidiano, sabemos que há um único ponto em ambas no plano euclidiano; o mesmo ponto estará em ambas as retas na nova estrutura proposta.
	\item Dadas duas retas \(\ell_1\) e \(\ell_2\) paralelas no plano euclidiano, o ponto do infinito relacionado a sua direção estará em ambas as retas na nova estrutura proposta.
	\item Dadas uma reta \(\ell_1\) no plano euclidiano e a reta \(\ell_\infty\) do infinito, o ponto do infinito relacionado a direção de \(\ell_1\) estará em ambas as retas na nova estrutura proposta.
\end{enumerate}

Portanto, essa nova estrutura é um plano projetivo, é é muito similar ao plano euclidiano --- de fato, é o plano euclidiano mais alguns pontos e uma reta. Chamamos essa estrutura de plano projetivo real.


	\part{Abordagem com Álgebra Linear}

	\chapter{Álgebra Linear}

Assumimos que o leitor esteja acostumado com conceitos relacionados à Álgebra Linear, especialmente em \(\mathbb{R}^2\) e \(\mathbb{R}^3\).
Definições e conceitos importantes incluem:
corpo, espaço vetorial, combinação linear, independência linear, espaços gerados, base de um espaço vetorial, dimensão de um espaço vetorial.
De qualquer modo, apresentaremos nesse capítulo um resumo desses conceitos.

\begin{defn}
    Falamos que o vetor \(\mathbf u\) é uma combinação linear entre os vetores \(\mathbf v_1\), \(\mathbf v_2\), \(\dots\), \(\mathbf v_n\), quando quando existem escalares \(\alpha_1\), \(\alpha_2\), \(\dots\), \(\alpha_n\) tais que
    \begin{equation}
        \mathbf u = \alpha_1\mathbf v_1 + \alpha_2\mathbf v_2 + \dots + \alpha_n\mathbf v_n.
    \end{equation}
\end{defn}

\begin{defn}
    Os vetores \(\mathbf v_1\), \(\mathbf v_2\), \(\dots\), \(\mathbf v_n\) são \emph{linearmente independentes} quando, se \(\alpha_1\), \(\alpha_2\), \(\dots\), \(\alpha_n\) são escalares satisfazendo
    \begin{equation}
        \alpha_1\mathbf v_1 + \alpha_2\mathbf v_2 +\dots + \alpha_n\mathbf v_n = 0,
    \end{equation}
    então
    \begin{equation}
        \alpha_1 = \alpha_2 = \dots = \alpha_n = 0.
    \end{equation}
\end{defn}

\begin{defn}
    O espaço gerado pelos vetores \(\mathbf v_1\), \(\mathbf v_2\), \(\dots\), \(\mathbf v_k\) é o conjunto de todas as possíveis combinações lineares entre eles, denotado por \(\langle\mathbf v_1,\mathbf v_2,\dots,\mathbf v_k\rangle\).
    Equivalentemente, é o menor subespaço vetorial contendo \(\mathbf v_1\), \(\mathbf v_2\), \(\dots\), \(\mathbf v_n\).
\end{defn}

\begin{defn}
    Se \(\mathbf v_1, \mathbf v_2, \dots, \mathbf v_n \in V\) são vetores linearmente independente tais que
    \begin{equation}
        \langle \mathbf v_1, \mathbf v_2, \dots, \mathbf v_n \rangle = V,
    \end{equation}
    então chamamos \(\{\mathbf v_1, \mathbf v_2, \dots, \mathbf v_n\}\) de \emph{base} de \(V\).
\end{defn}

\begin{thm}
    Dado um espaço vetorial \(V\), todas as suas bases possuem a mesma cardinalidade.
\end{thm}

\begin{defn}
    Chamamos de \emph{dimensão} de um espaço vetorial \(V\) o número de vetores em suas bases.
\end{defn}

	\chapter{Reta Projetiva}

Inicialmente definiremos o que são os pontos projetivos a partir do $\mathbb{R}^2$:

\begin{defn}[Ponto Projetivo]
Vamos definir uma relação de equivalência \(\sim\) entre vetores do $\mathbb{R}^2 \setminus \{(0, 0)\}$,
onde dizemos que \((a,b) \sim (c,d)\) se, e somente se, existe real não nulo \(\lambda\) tal que \[(a,b)=\lambda (c,d).\]

Denotaremos então um ponto projetivo na reta por sua classe de equivalência $[a,b]$ induzida pelo ponto euclidiano $(a,b)$.
\end{defn} 

Note portanto que, em termos de coordenadas, só estamos interessados na razão entre as coordenadas de um ponto euclidiano.

	\chapter{Plano Projetivo}

\section{Definição}

\begin{defn}[Plano Projetivo Real] \label{defn:planoprojetivoreal}
O \emph{plano projetivo real \(\mathbb{P}^2(\mathbb{R})\)} é o conjunto das retas do \(\mathbb{R}^3\) que passam pela origem.
Elementos do plano projetivo real são chamados de \emph{pontos projetivos}.
\end{defn}

Essa definição é completamente análoga à Definição \ref{defn:retaprojetivareal}; apenas aumentamos uma dimensão.

Como um abuso de notação, chamaremos o plano projetivo real somente de plano projetivo.
Quando o significado estiver claro, podemos chamar o plano projetivo de plano.

Em termos de Álgebra Linear, os pontos do plano projetivo são os subespaços do \(\mathbb{R}^3\) com dimensão~\(1\).

Podemos inferir uma representação dos pontos projetivos do plano projetivo a partir dos vetores do \(\mathbb{R}^3\), de modo completamente análogo ao que fizemos na Seção \ref{sec:defnretaprojetiva}.

Se \(\mathbf{v} \neq (0, 0, 0)\), vamos chamar de \([\mathbf{v}]\) a única reta que passa pela origem e por \(\mathbf{v}\); equivalentemente, também podemos definir \([\mathbf{v}]\) como \(\langle v \rangle\), o subespaço do \(\mathbb{R}^3\) gerado por \(\mathbf{v}\).
Podemos concluir que \([\mathbf{v}]\) é o conjunto dos múltiplos reais de \(\mathbf{v}\), isto é, é o conjunto \(\{ \lambda \mathbf{v} : \lambda \in \mathbb{R}\}\).

Portanto, temos novamente que \([\mathbf{v}] = [\mathbf{u}]\) se, e somente se, \(\mathbf{u}\) é um múltiplo real de \(\mathbf{v}\).
Em outras palavras, \([\mathbf{v}] = [\mathbf{u}]\) se, e somente se, existe \(\lambda \in \mathbb{R}\) tal que \(\mathbf{u} = \lambda\mathbf{v}\).

Por que chamamos o plano projetivo de ``plano''? Uma maneira de responder essa pergunta é, dada uma base do $\mathbb{R}^3$, considerar o plano $z=1$.
Com isso, podemos tomar como representante para cada ponto projetivo o vetor que leva da origem até sua interseção com o plano $z=1$.

Contudo, vemos que vários pontos projetivos não intersectam o plano \(z = 1\).
Mais precisamente, as retas que passam pela origem e estão contidas no plano \(z = 0\) são os pontos projetivos que não possuem representante pela regra acima.
Porém, o conjunto de pontos projetivos contidos em um plano real possui nome!
Chamamos esses conjuntos de reta projetiva.

Portanto, podemos interpretar a reta projetiva real como um plano real com uma reta projetiva extra.
Juntando com o que desenvolvemos na Seção \ref{sec:defnretaprojetiva}, podemos dizer que a reta projetiva real é um plano real, mais uma reta real, mais um ponto extra.
Podemos ver isso claramente em \ref{eqn:planoprojetivoplanoretaponto}.
\begin{equation} \label{eqn:planoprojetivoplanoretaponto}
  \mathbb{P}^2(\mathbb{R}) = \{ [x, y, 1] : x, y \in \mathbb{R} \} \cup \{[x, 1, 0] : x \in \mathbb{R}\} \cup \{[1, 0, 0]\}.
\end{equation}

\begin{exer}
  Convença a si mesmo de que a igualdade de conjuntos em \ref{eqn:planoprojetivoplanoretaponto} é verdadeira.
\end{exer}

\begin{thm} \label{thm:quatropontos}
  Dados quatro pontos projetivos distintos \(A\), \(B\), \(C\) e \(D\) podemos escolher uma base apropriada do \(\mathbb{R}^3\) para a qual \(A = [1, 0, 0]\),  \(B = [0, 1, 0]\), \(C = [0, 0, 1]\) e \( D = [1, 1, 1]\)
\end{thm}
\begin{proof}
Prova análoga ao teorema \ref{thm:trespontos} 
\end{proof}
\section{Colinearidade e concorrência}
\begin{defn}
Dado um plano que passa pela origem do $\RR^3$, os conjunto de todas as retas contidas nele que passam pela origem, é chamado de \emph{reta projetiva}.
\end{defn}

Outra forma de pensar a reta projetiva contida no plano projetivo é ver ela como o espaço projetivo derivado de qualquer subespaço vetorial de dimensão dois do $\RR^3$.

\begin{thm}
Dada uma base para o $\RR^3$ tal que os pontos do plano projetivo $A$,$B$ e $C$ tem coordenadas $[x_A,y_A,z_A], [x_B,y_B,z_B]$ e $[x_C,y_C,z_C]$ respectivamente. 
\[A,B \text{ e } C \text{ são colineares (pertencem a uma mesma reta)} \iff \det \begin{bmatrix} x_A & y_A & z_A \\ x_B & y_B & z_B \\ x_C & y_C & z_C \end{bmatrix} = 0\]
\end{thm}

\begin{proof}
Se os pontos são colineares os representantes pertencem ao mesmo plano logo eles são \textit{linearmente dependentes}, com isso temos que o determinante é zero.

Já se o determinante é zero os representantes são \textit{linearmente dependente} logo pertencem ao mesmo plano.
\end{proof}

Isso mostra que dado $a$ e $b$ representantes fixos para $A$ e $B$ todos os pontos projetivos que são colineares com eles podem ser escritos da forma $\alpha a + \beta b$ com $\alpha,\beta \in \RR$. Ou seja, tomando $a$ e $b$, como "bases" dessa reta, podemos dizer que os pontos dessa reta tem coordenadas $[\alpha,\beta]$ na reta, ou seja coordenadas de uma reta projetiva, isso nos permite calcular razão cruzada de pontos colineares no plano projetivo.

Opção 1

Por outro lado, sabemos que um há uma associação natural entre planos que passam na origem e retas que passam na oridem no $\RR^3$, que se baseia em para cada plano pegar o conjunto de vetores normal a ele, gerando a equação do plano como algo da forma $ax+by+cz = 0$, será que existe algo análogo para retas projetivas?

\begin{thm}
Dada uma base do $\RR^3$ e uma reta projetiva $r$, existem $a,b$ e $c$ tais que um ponto projetivo $P = [x,y,z]$ pertence a $r$ se, e somente se, $ax+by+cz=0$
\end{thm}
\begin{proof}
Depois eu escrevo
\end{proof}

Opção 2

Contudo, sabemos que um plano no $\RR^3$ pode ser descrito como, dada uma base, o conjunto de vetores $(x,y,z)$ tais que $ax+by+cz = 0$ com $a,b$ e $c$ reais fixos, mas se pegarmos um ponto projetivo qualquer, se um dos representantes dele pertence ao plano então todo o ponto também pertence, ou seja, temos que um ponto projetivo $P = [x,y,z]$ pertence a uma reta projetiva $r$ se, e somente se, vale que $ax+by+cz = 0$, sendo $a,b$ e $c$ relacionados a equação plano real associado $r$, com isso vamos dizer que as coordenadas da reta $r$ são $<a,b,c>$, note também que a coordenada das retas formam um sistema de coordenadas homogêneas pois $<a,b,c>$ e $<2a,2b,2c>$ representam o mesmo plano no $\RR^3$ logo a mesma reta no $\PP^2$

\begin{thm}[Primeiro Teorema de Desargues]
Dados dois triângulos $ABC$ e $A'B'C'$ as seguintes condições são equivalentes:
\begin{enumerate}
    \item $AA'$, $BB'$ e $CC'$ são concorrentes.
    \item $X = AB \cap A'B'$, $Y = AC\cap A'C'$ e $Z = BC \cap B'C'$ são colineares.
\end{enumerate}
\end{thm}

\begin{proof}
Vamos mostrar que $1 \implies 2$:

Seja $O = AA' \cap BB'$, pelo teorema \ref{thm:quatropontos} vamos tomar $A = [1, 0, 0]$, $B = [0, 1, 0]$, $C = [0, 0, 1]$ e $O = [1, 1, 1]$. Com isso temos que $AO = < 0, 1, -1>$, $BO = < 1, 0, -1>$ e $C) = < 1, -1, 0>$. Com isso temos que $A' = [a,1,1]$, $B' = [1,b,1]$ e $C'= [1,1,c]$, para alguns $a,b$ e $c$ reais. 

Agora vamos calcular $X$, $Y$ e $Z$. Note que: $AB = <0,0,1>$ e $A'B' = <b-1,a-1,1-ab>$, com isso $X = [a,-b,0]$, analogamente temos $Y = [-a,0,c]$ e $Z = [0,b,-c]$, como:

\[\det \begin{bmatrix} a & -b & 0 \\ -a & 0 & c \\ 0 & b & -c \end{bmatrix} = 0 \text{ temos que } X,Y \text{ e } Z \text{ são colineares } \]
\end{proof}

Agora para mostrar a volta vamos usar outro argumento.

Tome vamos chamar $BC$ de $a$, $AC$ de $b$ e $AB$ de $c$ e o análogo para $A'B'$,$A'C'$ e $B'C'$, vamos chamar $AA'$ de $x$, $BB'$ de $y$ e $CC'$ de $z$, por último vamos chamar $XY$ de $o$.

Note que a ida era, dados $ABC$ e $A'B'C'$, temos que se $AA'$, $BB'$ e $CC'$ concorrem então $AB \cap A'B'$, $AC\cap A'C'$ e $ BC \cap B'C'$ são colineares. Já a volta é temos, dados $abc$ e $a'b'c'$, se $a \cap a'$, $b \cap b'$ e $c\cap c'$ são colineares então as retas $a\cap b \ a'\cap b'$, $a\cap c \ a' \cap c'$ e $ b\cap c \ b'\cap c'$ são concorrentes.

Ou seja, a volta do problema é exatamente a ida "trocando" ponto por reta, e algo curioso do plano projetivo é que uma igualdade do tipo $ax+by+cz=0$ pode ser interpretada como, o ponto $[x,y,z]$ pertence a reta $<a,b,c>$ ou como o ponto $[a,b,c]$ pertence a reta $<x,y,z>$, com isso para resolver a volta a gente vai fazer o seguinte truque:

Escolha uma base qualquer para $\RR^3$, imagine para cada reta que você calcular as coordenadas nos pontos com os mesmos números como coordenadas, no mundo dos pontos nos já sabemos que a conta fecha, e como a conta é exatamente a mesma para um mundo com retas e um mundo com pontos, temos que no munda das retas a conta também fecha, ou seja a volta é uma consequência direta da ia.
\newpage

%parte para ser add no futuro
\begin{lem}[Lema Útil]
Se uma projetividade de uma $r$ em uma reta $s$ preserva um ponto então ela é uma projeção.
\end{lem}

\begin{thm}[Feixe de Projetividade]
Dada uma projetividade $\alpha$ de uma reta $r$ em outra $s$, seja $X = A\alpha(B) \cap B\alpha(A)$, temos que variando $A$ e $B$ em $r$, $X$ pertence a uma reta fixa, chamada \emph{feixe de $\alpha$}.
\end{thm}

\begin{cor}[Teorema de Pappus]
Dado $A$, $B$ e $C$ em uma reta e $A'$, $B'$ e $C'$ em outra, tome $X = AB' \cap A'B$, $Y = AC'\cap A'C$ e $Z = BC' \cap B'C$, temos que $X$, $Y$ e $Z$ são colineares.
\end{cor}


\begin{thm}[Segundo Teorema de Desargues]
Dado um quadrilátero $ABCD$ e uma reta fixa $r$, seja $X = AB \cap r$, $X' = CD \cap r$, $Y = AC \cap r$, $Y' = BD \cap r$, $Z = AD \cap r$ e $Z' = BC \cap r$, com isso temos que $(X,X')$, $(Y,Y')$ e $(Z,Z')$ são pares de uma involução na reta $r$.
    
\end{thm}
	\include{capitulos/projecoeseprojetividades}
	\chapter{Cônicas e Polaridades}


	\printbibliography

\end{document}
