\chapter{Cônicas e Polaridades}
\section{Cônicas degeneradas}
\begin{defn}
Uma cônica é chamada degenerada se o seu polinômio pode ser fatorado em duas retas distintas.
\end{defn}

\begin{thm}
Uma cônica é degenerada se, e somente se, o determinante da matriz relacionada à polaridade é zero.
\end{thm}
\begin{proof}
Seja $M = \begin{pmatrix} a& d& e \\ d& b& f \\ e& f& c\end{pmatrix}$ a matriz da polaridade da cônica.

Vamos mostrar que se o polinômio é fatorável então o determinante é zero.
Suponha que $P(x,y,z) = ax^2+by^2+cy^2+2dxy+2exz+2fyz = (mx+ny+lz)\cdot(m'x+n'y+l'z)$. Com isso,

\begin{align*}
    a &= mm' \\
    b &= nn' \\
    c &= ll'\\
    2d &= nm'+n'm\\
    2e &= lm'+l'm\\
    2f &=ln'+l'n.
\end{align*}
Como $\det(M) = abc + 2def -e^2b-d^2c-f^2a$, substituindo a parametrização vemos que $\det(M) = 0$.

Agora vamos mostrar que se o determinante é zero então é fatorável.

Para isso vamos abrir em casos:
\begin{enumerate}
    \item $a=b=c=0$.
    \item Algum deles diferente de 0, sem perda de generalidade o $a$.
\end{enumerate}

Caso $a=b=c=0$, temos que $\det(M) = 2def = 0$. Sem perda de generalidade, $d=0$, ou seja:
\[ P(x,y,z) = 2exz+2fyz = 2\cdot(x+0\cdot y + 0 \cdot z)(0\cdot x + y + z).\]
Logo, $P(x,y,z)$ é fatorável.

Caso $a \ne 0$.
Considerando o $P(x,y,z)$ uma quadrática em $x$, temos que ele pode ser escrito como

\[ \left(x - \frac{(-dy - ez + \sqrt{\Delta_x})}{a}\right)\left(x - \frac{(-dy - ez - \sqrt{\Delta_x})}{a}\right),\]
com 

\[\Delta_x =-aby^2 - acz^2 - 2afyz + d^2y^2 + 2deyz + e^2z^2.\]

Se o $\Delta_x$ for quadrado perfeito como um polinômio teremos que é possível escrever $P(x,y,z)$ como produto de dois polinômios de grau $1$. Agora vamos dividir em mais dois casos:

\begin{enumerate}
    \item $d^2 = ab$ e $e^2 = ac$.
    \item Algum dentre $d^2 - ab$ e $e^2 - ac$ é não nulo. 
\end{enumerate}
Se $d^2 = ab$ e $e^2 = ac$, 

\[0 = \det M = -f^2a + 2def - abc\]
\[\implies f = \dfrac{2de\pm \sqrt{4d^2e^2-4a^2bc}}{2a}\]
\[\implies f = \dfrac{de}{a}\]
\[\implies af = de\]
\[\therefore \Delta_x = 0.\]

Agora, sem perda de generalidade, suponha $d^2\ne ab$. Então, $\Delta_x$ é uma quadrática em $y$, sendo o $\Delta$ dessa quadrática igual a 
\[-a^2bc + a^2f^2 + abe^2 + acd^2 - 2adef = -a\cdot\det(M) = 0,\]
logo o $\Delta_x $ é um trinômio do quadrado perfeito em $y$, portanto $P(x,y,z)$ é fatorável.
\end{proof}