\chapter{A reta real não tem harmonia}

\section{Divisão Harmônica}

\begin{exer}
  Dados dois pontos \(A\) e \(B\) distintos na reta real, marque, com régua e compasso, todos os pontos \(P\) na reta real tais que \(\frac{AP}{BP} = 2\).
\end{exer}

Você observará que existem dois pontos, digamos \(P\) e \(Q\) --- um dentro do segmento \(AB\) e o outro fora do segmento \(AB\) ---  tais que \(\frac{AP}{BP} = \frac{AQ}{BQ} = 2\).

Por terem essa propriedade, dizemos que \(P\) e \(Q\) dividem \(A\) e \(B\) harmonicamente, na razão \(2\).

% Insira a figura A---P-------B-----------Q.

Não há nada especial com o número \(2\); \emph{parece} que, dado qualquer \(\lambda\) real positivo, existem sempre dois pontos que dividem \(A\) e \(B\) harmonicamente na razão \(\lambda\)...
Na verdade, tem um valor de \(\lambda\) para o qual isso não é verdade.

\begin{exer}
  Dados dois pontos \(A\) e \(B\) distintos na reta real, marque, com régua e compasso, todos os pontos \(P\) na reta real tais que \(\frac{AP}{BP} = 1\).
\end{exer}

Você observará que somente um ponto possui a propriedade acima; o ponto médio do segmento \(AB\).
Não existe ponto fora do segmento \(AB\) com essa propriedade --- de fato, um dos segmentos \(AP\) ou \(BP\) estará contido no outro, e portanto um será menor que o outro.

