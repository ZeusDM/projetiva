\chapter{Reta Projetiva}

Inicialmente definiremos o que são os pontos projetivos a partir do $\mathbb{R}^2$:

\begin{defn}[Ponto Projetivo]
Vamos definir uma relação de equivalência \(\sim\) entre vetores do $\mathbb{R}^2 \setminus \{(0, 0)\}$,
onde dizemos que \((a,b) \sim (c,d)\) se, e somente se, existe real não nulo \(\lambda\) tal que \[(a,b)=\lambda (c,d).\]

Denotaremos então um ponto projetivo na reta por sua classe de equivalência $[a,b]$ induzida pelo ponto euclidiano $(a,b)$.
\end{defn} 

Assim, estamos fazendo uma correspondência entre um ponto projetivo e uma reta que passa pela origem do $\mathbb{R}^2$.

Note que, em termos de coordenadas, só estamos interessados na razão entre as coordenadas de um ponto euclidiano (exceto quando uma delas é zero, o que é fácil de lidar). 
